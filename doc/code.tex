The code used for this project was written using C++.  It was broken into five distinct pieces in order to maintain organization and at least some level of abstraction.  Each of these pieces will be described in sections \ref{ss:utils}-\ref{ss:driver}.

\subsection{Utils}\label{ss:utils}

This module contains functions which will be needed throughout the code and are not tied to any specific functionality.  Some of these functions are as follows:

\begin{itemize}
\item \textit{double drand(void)} -- This function uses the intrinsic C function \textit{int rand()} to return a random double precision number between 0.0 and 1.0.
\item \textit{double Watt(void)} -- This function returns a random energy for a fission neutron, sampled from the Watt spectrum.
\item \textit{bool approxeq(double, double)}, \textit{bool approxge(double, double)}, \textit{bool approxle(double, double)} -- These functions are used for $==$, $\ge$, and $\le$ operators, respectively, for floating point double precision numbers.
\end{itemize}

In addition to these functions, the header file for Utils also defines many constants for use in the code, such as pi, the mass of a neutron, the Boltzmann constant, and others used for various parts of the code.

\subsection{Geometry}\label{ss:geometry}



\subsection{Materials}\label{ss:materials}



\subsection{Particles}\label{ss:particles}



\subsection{Driver}\label{ss:driver}



